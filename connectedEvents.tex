% THIS IS SIGPROC-SP.TEX - VERSION 3.1
% WORKS WITH V3.2SP OF ACM_PROC_ARTICLE-SP.CLS
% APRIL 2009

\documentclass{acm_proc_article-sp}

\newcommand{\superscript}[1]{\ensuremath{^{\textrm{#1}}}}
\def\sharedaffiliation{\end{tabular}\newline\begin{tabular}{c}}

\def\wu{\superscript{1}}
\def\wg{\superscript{2}}

\begin{document}

\title{Context-aware Connections between News Events}
%\subtitle{Exploring DBpedia paths between named entities belonging to different event contexts}
% MOVED TO ABSTRACT TO SAVE SPACE

\numberofauthors{5} 

\author{
\alignauthor
Laurens De Vocht\wu
       \affaddr{\email{\texttt{laurens.devocht@ugent.be}}}
\and
\alignauthor
Erik Mannens\wu
       \affaddr{\email{\texttt{erik.mannens@ugent.be}}}
\and
\alignauthor
Rik Van de Walle\wu
       \affaddr{\email{\texttt{rik.vandewalle@ugent.be}}}
% 3rd. author
\and
\alignauthor 
Rapha\"el Troncy\wg
	\affaddr{\email{\texttt{raphael.troncy@eurecom.fr}}}
\and
\alignauthor 
Jos\'e Luis Redondo Garc\'ia\wg
	\affaddr{\email{\texttt{redondo@eurecom.fr}}}
\sharedaffiliation
\begin{tabular}{ccc}
    \affaddr{{\wu}Ghent University - iMinds - Multimedia Lab{\ }} & & 
    \affaddr{{\wg}EURECOM{\ }} \\
    %\affaddr{Gaston Crommenlaan 8/201} & \affaddr{} \\
    \affaddr{Ghent, Belgium} & &
    \affaddr{Biot, France} \\
\end{tabular}
}
%USED SHARED AFFILIATIONS TO SAVE SPACE

\maketitle
\begin{abstract}
Exploring DBpedia paths between named entities belonging to different event contexts...

As extracting named entities out of media is more common, many applications and users want to leverage these entities to construct and link to other media.

\end{abstract}

% A category with the (minimum) three required fields
%\category{H.4}{Information Systems Applications}{Miscellaneous}
%A category including the fourth, optional field follows...
%\category{D.2.8}{Software Engineering}{Metrics}[complexity measures, performance measures]

%\terms{Theory}

%\keywords{ACM proceedings, \LaTeX, text tagging} % NOT required for Proceedings

\section{Introduction}

Online media is increasing in scale and ubiquity, but it is currently still unstructured and not good connected to media of other forms or from other sources.

LinkedTV is an integrated and practical approach towards experiencing Networked Media in the Future Internet.

[More about LinkedTV?]

Within LinkedTV we want to connect media based on extracted entities that we link to DBpedia resources.

[State of the Art: Named Entity Extraction, Expansion]

We uses therefore an optimized pathfinding algorithm \cite{de2013discovering} implemented in the Everything is Connected Engine (EiCE), firstly introduced with the Everything is Connected (EiC) Demo at the \emph{ISWC`12 Boston} conference.

[State of the Art: Everything is Connected]

\section{The Approach}

Given events expressed in documents (textual dimension). 
World wide coverage.

\subsection{Context through Expansion}



\subsection{Generating DBpedia paths}

INPUT: A main input item, a destination and VIA Points

[TODO Laurens]

Before we filter the relations between resources, the candidate resources to be included in relations are being pre-ranked. They are pre-ranked according to ``popularity'' and ``rarity'' essential components in the original PageRank algorithm \cite{page1999pagerank} and is used to sort candidate related nodes in the EiCE. The implementation of the EiCE takes the relations in to account by making use of the Jaccard coefficient to measure the dissimilarity and assign random walks based weight able to highly rank more rare resources, guaranteeing that paths between resources prefer specific relations and not general ones \cite{moore2012novel}.

OUTPUT: A set of optimized paths used to for the context expansion

\subsection{Filtering Context Relevant Paths}

TODO: Ranking mechanism

Count(?) the most frequently(?) occurring:

predicate,
resource,
query pattern





\subsection{Results}



\section{Use Case: Snowden Assylum}


\section{Conclusions}
We presented a new approach for context-aware connecting news events. Our preliminary results indicate that by exploring DBpedia paths in named entities occuring in news media. ...
%\end{document}  % This is where a 'short' article might terminate

%ACKNOWLEDGMENTS are optional
\section{Acknowledgments}
The research activities described in this paper were funded by Ghent University,
%iMinds (Interdisciplinary institute for Technology) a research institute founded by the Flemish Government,
the Institute for the Promotion of Innovation by Science and Technology in Flanders (IWT), the Fund for Scientific Research-Flanders (FWO-Flanders), and the European Union.

[EURECOM related acknowledgement]

%
% The following two commands are all you need in the
% initial runs of your .tex file to
% produce the bibliography for the citations in your paper.
\bibliographystyle{abbrv}
\bibliography{connectedEvents}  % sigproc.bib is the name of the Bibliography in this case
% You must have a proper ".bib" file
%  and remember to run:
% latex bibtex latex latex
% to resolve all references
%
% ACM needs 'a single self-contained file'!
%

\end{document}
